\documentclass[a4paper,11pt,twoside]{article}
%\documentclass[a4paper,11pt,twoside,se]{article}

\usepackage{UmUStudentReport}
\usepackage{verbatim}   % Multi-line comments using \begin{comment}
\usepackage{courier}    % Nicer fonts are used. (not necessary)
\usepackage{pslatex}    % Also nicer fonts. (not necessary)
\usepackage[pdftex]{graphicx}   % allows including pdf figures
\usepackage{listings}
%\usepackage{lmodern}   % Optional fonts. (not necessary)
%\usepackage{tabularx}
%\usepackage{microtype} % Provides some typographic improvements over default settings
%\usepackage{placeins}  % For aligning images with \FloatBarrier
%\usepackage{booktabs}  % For nice-looking tables
%\usepackage{titlesec}  % More granular control of sections.

% DOCUMENT INFO
% =============
\department{Institution för Datavetenskap}
\coursename{Datavetenskapens byggstenar 7.5 p}
\coursecode{DV160HT15}
\title{OU4 Analysis of Complexity}
\author{Lorenz Gerber  ({\tt{dv15lgr@cs.umu.se}})}
\date{2015-12-25}
%\revisiondate{2015-09-15}
\instructor{Lena Kallin Westin / Johan Eliasson}


% DOCUMENT SETTINGS
% =================
\bibliographystyle{plain}
%\bibliographystyle{ieee}
\pagestyle{fancy}
\raggedbottom
\setcounter{secnumdepth}{2}
\setcounter{tocdepth}{2}
%\graphicspath{{images/}}   %Path for images

\usepackage{float}

\floatstyle{ruled}
\newfloat{program}{thp}{lop}
\floatname{program}{Program}


% DEFINES
% =======
%\newcommand{\mycommand}{<latex code>}

% DOCUMENT
% ========
\begin{document}
\lstset{language=C}
\maketitle

\tableofcontents
\newpage

\section{Introduction} 
The aim with this laboration was to apply experimental and asymptotic
complexity analysis of algorithms. 

What is complexity analysis. Experimental, asymptotic. What is big O
notation, what does it mean.
 
\cite[pp. 117 -- 132]{janlert2000}.

\section{Material and Methods}

\subsection{Experimental Complexity Analysis}
Describe experiment, describe the rules.

\subsection{Asymptotic Complexity Analysis}
Describe what was given and the rules to analyse.
\begin{program}
\begin{verbatim}
Algorithm bubbleSort(numElements, list[])
input:  numElements, the number of elements in the list
        list, a list of numbers to be sorted
output: the sorted list

1:  done <- false
2:  n <- 0
3:   while (n < numElements) and (done = false)
4:      done <- true
5:      for m <- (numElements -1) downto n
6:          if list[m] < list[m - 1] then
7:              tmp <- list[m]
8:              list[m] <- list[m - 1]
9:              list[m - 1] <- tmp
10:             done <- false
11:    n <- n + 1
12: return list 
\end{verbatim}
\caption{This is the shit}
\end{program}

\section{Results}
\subsection{Experimental Complexity Analysis}
Show formulas, C, n0

\subsection{Asymptotic Complexity Analysis}
\subsubsection{Worst Case}
\begin{verbatim}

1:  1 * [done <- false] + 
2:  1 * [n <- 0] + 
3:  (numElements + 1) * ([n < numElements] + [done = false]) + 
4:  numElements * [done <- true] + 
5:  ((numElements(numElements+1)) / 2) * (1 * [m <-] + 1 * [numElements - 1]
6:  1 * [> n] + $\\
7:  1 * [list[m]] + 1* [m-1] + 1 * [list[]] + 1 * [<] + 
8:  1 * [m-1] + 1 * [list[]] + 1 * [<-] + 
9:  1 * [tmp] + 1 * [m-1] + 1 * [list[] <-] +
10: 1 * [done <- false] ) +
11: 1 * [n + 1] + 1 * [n <-] +
12: 1 * return

set numElements = x

1 + 1 + (x + 1) * 2 + x + (x(x+1) / 2) * ( 1 + 1 + 1 + 1 + 1 + 1
+ 1 + 1 + 1 + 1 + 1 + 1 + 1 + 1) + 1 + 1 + 1
2 + 2x + 2 + x + (1 / 2x^2} + {1 / 2x) * 14 + 3
3x + 6 + 7x^2 + 7x
7x^2 + 10x + 7
\end{verbatim}

\subsubsection{Best case}
\begin{verbatim}
1: 1 * [done <- false] +
2: 1 * [n <- 0] +
3: 2 * ([n < numElements] + [done = false]) +
4: 1 * [done <- true] +
5: (numElements - 1) * (1 * [m <-] + 1 * [numElements - 1] + 1 * [> n])

11: 1 * [n+1] + 1 * [<-]
12: 1 * [return]

set numElements = x

1 + 1 + 2 * 2 + 1 + (x-1) * (1 + 1 + 1) + 1 + 1 + 1
10 + 3x - 3
3x + 7
\end{verbatim}

show plot

\section{Discussion}
\subsection{Experimental Complexity Analysis}

\subsection{Asymptotic Complexity Analysis}


\addcontentsline{toc}{section}{\refname}
\bibliography{references}

\end{document}
